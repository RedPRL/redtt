\documentclass{article}

\usepackage{fullpage}
\usepackage{jst}
\usepackage{mathpartir}
\usepackage{notation}

\begin{document}

Note that as a matter of convention, I write $\Meta{s}{\DimDelete}$ to mean
``delete this tube if the resulting instantiation mentions $\DimDelete$'' (it's
how I implement $\forall$).

\[
  \inferrule[CoeFcom/0]{
    \Alert{
      \Eq{
        \Meta{s}{r'}
      }{
        \Meta{s'}{r'}
      }
    }
    \\
    \NotEq{
      \Meta{s}{x}
    }{
      \Meta{s'}{x}
    }
    \\
    \NotEq{
      \Meta{s_i}{x}
    }{
      \Meta{s'_i}{x}
    }
  }{
    \BreakStep{
      \Coe{r}{r'}{x}{
        \FCom{
          \Meta{s}{x}
        }{
          \Meta{s'}{x}
        }{
          \Meta{A}{x}
        }{
          \Vec{
            \BTube{
              \Meta{s_i}{x}
            }{
              \Meta{s'_i}{x}
            }{z}{
              \Meta{B_i}{x\;z}
            }
          }
        }
      }{M}
    }{
      \GCom{r}{r'}{x}{
        \Meta{A}{x}
      }{M}{
        \Vec{
          \BTube{
            \Meta{s_i}{\DimDelete}
          }{
            \Meta{s'_i}{\DimDelete}
          }{x}{
            \Coe{
              \Meta{s'}{x}
            }{
              \Meta{s}{x}
            }{z}{
              \Meta{B_i}{x\;z}
            }{
              \Coe{r}{x}{x}{
                \Meta{B_i}{x\;\Meta{s}{x}}
              }{M}
            }
          }
        }\;
        \BTube{
          \Meta{s}{\DimDelete}
        }{
          \Meta{s'}{\DimDelete}
        }{x}{
          \Coe{r}{x}{x}{
            \Meta{A}{x}
          }{M}
        }
      }
    }
  }
\]

It is not at all clear how to implement the evaluation semantics for such a
rule, at least in a na\"ive way. That is, in the presentation above we maintain
the fiction that we can ``see'' into the binder-closures well enough to take
out the $\Meta{A}{x}$ (or even the $\Meta{s}{x}$!), etc.\ and use it elsewhere.
We do not have this luxury, however, in the semantic domain; generally, the
closures need to be thought of as black boxes.

One idea that doesn't work is to add a new kind of stack frame to the binder
closure, that says \emph{During instantiation, if this becomes an $\MathKwd{fcom}$,
project the cap!} But this doesn't work, since depending on with what
dimension the closure is instantiated, the $\MathKwd{fcom}$ may disappear, and
(for instance) be replaced by one of its tubes. Then, we have lost track of the
cap forever.

However, I think there is a way out. During evaluation, when we encounter a
non-rigid composition, we currently project out the appropriate part (either the
cap or one of the tubes), and lose the rest of the information. We could instead \emph{not} do this, but actually just annotate the composition with the value that would have been projected out.

Then, meta-operations like semantic function application would see and use this
projection; likewise, quotation and definitional equivalence would do the same.
However, by keeping this information, we would be able to extend the language
of closures with a stack frame that can project the cap from an
$\MathKwd{fcom}$, even if at instantiation-time it would have already evaluated
to the tube. In particular, this allows us to implement a continuation which
grabs the appropriate part of the composition for use in the resulting
$\MathKwd{gcom}$.

\end{document}
